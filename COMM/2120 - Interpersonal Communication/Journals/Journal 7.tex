%Chapter seven discussed eight (8) types of nonverbal communication: body movement (incorporates face and eyes, posture, and gestures), touch, voice, distance, territoriality, time, physical attractiveness, and clothing.

%   Give a personal example of how you have used each type of nonverbal communication. *Notes: You should have a total of eight examples. Each example should reflect specific incidents, not just general tendencies.

%    Based on these examples, personal reflection, and comments made to you by others, rate yourself as a nonverbal communicator in terms of frequency—how often you use nonverbal communication.

%    Based on these examples, personal reflection, and comments made to you by others, rate yourself as a nonverbal communicator in terms of awareness. (Do you realize you are communicating nonverbally?)

%    Based on these examples, personal reflection, and comments made to you by others, rate yourself as a nonverbal communicator in terms of accuracy of encoding and decoding messages—how accurate you are when sending nonverbal messages and how accurate you are when receiving nonverbal messages from others.

\documentclass[12pt]{article}
\usepackage{times}
\usepackage{setspace}
\doublespacing
\usepackage[margin=1in]{geometry}
\begin{document}
\title{Reflection Journal 7}
%   Give a personal example of how you have used each type of nonverbal communication. *Notes: You should have a total of eight examples. Each example should reflect specific incidents, not just general tendencies.
\par
I can think of multiple ways I have used nonverbal communication. The first one I can think of is shrugging with eyebrows raised, communicating "I recognize this job is done poorly, and there are far superior ways to rectify the problem, however it is beyond our control". This was done very recently in a conversation with my friends Matthew and Tom, where we were discussing extreme security holes that are worthy of legal action within the school system, as well as the IT department's refusal to learn anything new.
\par
The second form of nonverbal communication I use frequently, is the widening of my eyes to express emphasis, surprise, or disbelief depending on context. The most recent time I remember doing this was when an ENMU Softball coach complained that I was being snippy with her and complained to my supervisor, yet all I have done was try to state that a product was supposed to go somewhere else so they have room for hot dogs.
\par
The third example would be yawing my head to the right, raising my left eyebrow, and squinting slightly which represents me not believing anything being said by the person talking to me, an interrogative reaction. I performed this last weekend when my friend Andrew was stating that his friend put bluetooth on buoys in the ocean for long range communication, which bluetooth itself is a very short range personal area network. Even with long range antennas and high wattage radios, the concept still does not check out.
\par
The fourth would be raising my left hand and resting on my chin and mouth, whilst bouncing my right leg if needed to express that I am pondering something in a communicational transaction. For example, my friend was talking to me about ways we can improve our Computer Science club at the University, and I subconsciously went to that position.
\par
The fifth would be when driving, I roll my window down, stick my hand out above my roof, and point to the right to tell a driver to get into the cruising lane. I did this recently with a driver that disallowed me into the passing lane nearly causing an accident when I had to swerve to the passing lane quickly; this nonverbal communication was done out of necessity since it was a safety hazard and illegal in this state.
\par
The sixth nonverbal communication would be me squinting my eyes and jerking my head forward. This expresses "Think about it", which I have used with my friend Matthew about a way of network booting an operating system for a computer, and attempting to drive home more critical thought in a particular area.
\par
The seventh form of my nonverbal communication would be dropping my clutch in my car and revving my engine, which depending on the context can mean "Watch how you're driving" or "Nice to see you!".
\par
My final form of nonverbal communication would be catching someone's eyes and then flicking my eyes in the direction I want them to pay attention to. The most recent time I have done this would be when I was trying to communicate a problem volunteer with my supervisor without directly mentioning their name nor pointing out their direction with my arm.
%    Based on these examples, personal reflection, and comments made to you by others, rate yourself as a nonverbal communicator in terms of frequency—how often you use nonverbal communication.
\par
I haven't recieved any comments by others on my nonverbal communicational skills, but whenever I perform a nonverbal communication, especially the examples that I have provided, the person I am attempting to communicate with typically executes the task immediately so I would state that my nonverbal communicational skills are very effective.
%    Based on these examples, personal reflection, and comments made to you by others, rate yourself as a nonverbal communicator in terms of awareness. (Do you realize you are communicating nonverbally?)
\par
I would absolutely say I realize that I am communicating nonverbally since the vast majority of the examples that I have provided are concious communications.
%    Based on these examples, personal reflection, and comments made to you by others, rate yourself as a nonverbal communicator in terms of accuracy of encoding and decoding messages—how accurate you are when sending nonverbal messages and how accurate you are when receiving nonverbal messages from others.
\par
For recieving nonverbal cues, I mostly understand what the person is attempting to communicate, though I have missed a few cues. And again, I feel that I am precise in communicating nonverbally since I don't remember a case of a nonverbal cue not being executed by the party I am attempting to communicating with. Thus, I would state I am a high performer at understanding and performing nonverbal cues.
\end{document}
