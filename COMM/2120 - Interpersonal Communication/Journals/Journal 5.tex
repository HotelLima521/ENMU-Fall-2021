\documentclass[12pt]{article}
\usepackage{times}
\usepackage{setspace}
\doublespacing
\usepackage[margin=1in]{geometry}
\begin{document}
\title{Reflection Journal 5}

%Review the section titled “Common Tendencies in Perception” (pgs. 134-139).
%Provide an example of a time in your life when you have experienced each tendency. (You should have six examples, one example per tendency). Within each example, answer the following:
%What were the outcomes?
%Would the outcomes be different if that tendency had not played a part in your perception?

%We make Snap Judgements
\par
%We cling to first impressions
\par
Clinging to first impressions is a perception that I continue to struggle with my landlord on. I still perceive her as a sweet old lady that is very accommodating, where in reality she judges quite harshly and is voraciously racist. She has caught me off guard talking about various races or other religions that weren't white or Christian in a completely inappropriate way for \emph{any} audience. The outcomes of this situation was a complete shock out of the things that she has said and me beginning to reevaluate my living situation as her tenant. The outcome would potentially be different if I \emph{knew} she acted this way, since I would more than likely never have met her-- and would have likely never become her tenant. 
%We Judge ourselves more charitably than we do others
\par

%We are influenced by our expectations
\par
The most recent time I have been influenced by my expectations wasn't with a person per se, however was with a video card for my computer. I was frustrated with video drivers from Nvidia, and decided that I needed to make a change to AMD since their drivers were open source instead of proprietary; allowing the drivers for the card itself to be implemented directly into my kernel and the 3D acceleration drivers built into the Mesa package, which is an open source collection of 3D acceleration drivers. The outcomes of this perception was a significant improvement in quality of life since I use my PC so often, and not having to worry about proprietary drivers breaking my system helped out a lot. If the outcomes were different, and I would still be using Nvidia as my video card, I would still be dealing with a ton of driver issues on my system.
%We are influenced by the obvious
\par
I have been influenced by the obvious recently. My friend has been struggling with having a daughter, and recently broke up with his girlfriend whilst trying to make enough money to provide for his daughter. I have been waiting half a year for him to finish working on my Jeep so it will be winter ready, and practically no work has been done on it. My perception of him was he is just too lazy to get it done, whereas in reality he is far too busy, and far too focused to obtain enough money to stay afloat that I am the least priority at the moment. The outcomes of this perception was frustration on my end for not having a winter-ready vehicle available. The outcomes of this perception would be different because I would have most likely just brought my Jeep to another friend to work on it for me so it would be handled on time.
%We assume others are like us
\par
Assuming others are like us is a perception I run into all too often. Once I get good enough at a skill set, I forget what others don't know and assume they are at a similar skill level to me as if it's common knowledge. I feel like this happens the most when we begin to devote a lot of our time to specific skill sets, for example vehicle mechanics, Information Technology, Construction, etc. You learn in depth in something super specific and forget what people \emph{don't} know. This has happened to me quite a few times, I worked in a specialized field in the Air Force where we would have to setup satellite communications on various air frames, and when other radio or network technicians came in they would be confused between either having to learn networking, having to learn radio theory, or having to learn some server management skill sets. The outcome was both parties (them and I) being equally frustrated at the scenario as we had to learn how to properly communicate the question and the answer. The outcome would be significantly different if this tendency had not played a part in this perception, since I would have known to start from the beginning relatively speaking, and ease the new technician into the environment.



\end{document}
