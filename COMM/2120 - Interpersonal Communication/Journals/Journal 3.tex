\documentclass[12pt]{article}
\usepackage{times}
\usepackage{setspace}
\doublespacing
\usepackage[margin=1in]{geometry}
\begin{document}
\title{Reflection Journal 1}

% Review the section titled “Developing Intercultural Communication Competence” (pgs. 78-83) then answer the following:

%    Give an example of a time when you engaged in intercultural communication and the interaction was successful.
%    What components of intercultural communication competence made the exchange a positive one?
%    Give an example of a time when you engaged in intercultural communication and a misunderstanding occurred.
%    What components of intercultural communication competence were missing?



\par
%    Give an example of a time when you engaged in intercultural communication and the interaction was successful.
A time that I engaged in an intercultural communication and the interaction was successful was between my friend Laura and I. She is Suomi, or Finnish and her culture is quite a bit different in comparison to mine. Their culture is quite shy and I would say pretty independent. However, her and another Finnish friend of mine, Jaakko really love talking about food that we have cooked, or cocktails that we have made for some reason. Perhaps food is a phenomenal way to cross-communicate interculturally? 
\par
%    What components of intercultural communication competence made the exchange a positive one?
The component of intercultural communication competence that made this exchange a positive one would be \emph{knowledge and skill}. Each of us had to recognize the skill sets required in making the food, or the drinks and were fascinated by it- even longing to try them. Thus, the connection was absolutely positive.
\par
%    Give an example of a time when you engaged in intercultural communication and a misunderstanding occurred.
It is difficult to place a specific time that I engaged in an intercultural communication with a misunderstanding. I am sure that it has happened multiple times before, but I don't think that it was significant enough to remember. The one time that I \emph{do} remember a cultural misunderstanding sticking out to me was when I went on vacation to Mexico as a little kid and I tried to order a hot dog from a street food vendor that only spoke Spanish. I was confused, he was confused, I needed parental assistance.
\par
%    What components of intercultural communication competence were missing?
The component that was missing in this case \emph{also} happens to be \emph{knowledge and skill} since I was so young at the time that I did not know other cultures spoke different languages. I thought that hot dog just meant hot dog, and we all spoke one uniform langage. It was an eye opening experience for me and it's one of those thoughts that continue to pop into my head as my mother came to rescue me from the awkward social encounter of not knowing how to communicate.
\end{document}
