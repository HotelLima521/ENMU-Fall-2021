\documentclass[12pt]{article}
\usepackage{times}
\usepackage{setspace}
\doublespacing
\usepackage[margin=1in]{geometry}
\begin{document}
\title{Reflection Journal 9}
%    Describe a time when you experienced emotional contagion.
I personally experience emotional contagion frequently. The most recent time would be when I was going out to coffee with a friend in one of my classes, I was in a relatively hyper mood. As soon as I was in her presence, I seemed to automatically slow down and parallel with the relaxed, calm emotions she was giving off to be in an emotional synchronization. Another time I remember quite clearly was when my friend's boyfriend died, and I was calling them. They began crying and then immediately I did as well.
%    What strategies do you use to control your emotions, despite what others around you are feeling?
To control my emotions I \emph{have} to be alone, especially once I start becoming frustrated or angry. I must have the time to decompress or I'll burst out in raw emotions which no one wants to be around. All of the other emotions I feel are fine in their raw state, and can even be welcomed depending on how \emph{much} energy I output. For example, getting too excited in a social situation due to over-stimulation can lead to becoming extremely annoying which I have a bad habit of doing-- and I know I've already done multiple times in this class. It can also be talking too long with someone and then noticing that they aren't as interested in talking. Reading the social cues either in person or over mediated communication leads me to re-adjust myself to become more in sync with the emotions of other people involved.

%    Give an example of a time when you or someone you know did not express emotions effectively. What was the outcome of that ineffective expression?
Roughly a month ago, I messaged my friend before class started. I told her an interesting squirrel fact, where they can survive a fall at terminal velocity, which is the fastest speed an object of its relative aerodynamics and density can travel in atmosphere. I mentioned that if you drop a squirrel from the edge of Earth's gravitational pull, discount the possibility of a squirrel burning up in re-entry, or dying from asphyxiation, the squirrel will then land perfectly fine back on the ground. She messaged me back angrily asking why I was messaging her at 9am, and if I wanted anything from her. As it turns out, she was frustrated at her Professor for breaking a HIPPA law, talking about her disability in front of the entire class without her consent as she walked in to take an exam. She later apologized for it, but I went into an emotional shutdown and felt uncomfortable around her that day.

%        Which suggestions for expressing emotions effectively (recognize your feelings, choose the best language, share multiple feelings, recognize the difference between feeling and acting, accept responsibility for your feelings, and choose the best time and place to express your feelings) were not taken into consideration in the example above.

\par
Choosing the best language and choosing the right time and place to express their feelings were not taken into consideration when my friend acted upon her feelings. Waiting a bit longer, or even stating that she was frustrated at someone else and that she wants to talk about how frustrated she is when I see her at the start of the next class would have made me feel included with her frustration, and be part of her team instead of feeling outcast a bit and unwanted.

\end{document}
