\documentclass[12pt]{article}    
\usepackage{times}    
\usepackage{setspace}    
\doublespacing    
\usepackage[margin=1in]{geometry}    
\begin{document}    
\title{Reflection Journal 12}

%    Fill out the “Conflict Styles Survey" CONFLICT STYLES SURVEY.docx  
%    Report the scores you determined in each of the five categories.
%    Analyze the results of the survey. Do you agree that the scores reflect your style(s) in managing conflict? Use specific personal examples to support your conclusions.
\par
I definitely agree with most of my results. I definitely see myself preferring Collaboration when in conflict, especially if it's with someone I don't have bad blood with. The way I view it is, we clearly have a problem, let's talk about our issues that we have so we can work through the issue and prevent this from happening in the future-- that way, both of us can set our boundaries to ensure they are not stepped over again. For Competition, I can definitely see this being my lowest score. Arguments should not be a competition, where it then becomes a "dick measuring contest", it's completely unproductive and does more harm than good. However, the highest score I gave on that category, \emph{Put your foot down where you mean to stand}, I took more so as standing your ground on boundaries or actual laws that have been broken. I don't consider that as competition whatsoever. For Compromising and Avoidance, I can see those being somewhat high scores, since I try to make an effort to either avoid a situation that I know is going to be bad and cannot be fixed by the other party due to lack of effort, or in the case of Compromising, I try to show that I care about the relationship even if it's in vain. It allows me to sleep through the night. For Accommodating, I can see this being my second lowest score because it seems like such a fake way of trying to get out of a conflict; you aren't being honest with yourself in my opinion and I think makes you appear stepped over.
%    Identify ways you can improve the ways you approach and manage conflicts.
\par
I think that the way I currently try to manage conflicts is working. I explain my thought process, and when I realize I am at fault I apologize. If I understand that I'm not the one at fault, I try my best to help educate the person so the conflict does not escalate. However, this is extremely difficult to do with people who don't care about what you are trying to state, nor care about your opinion which I don't have an answer for managing the conflict to improve it. Then if I was to move to Accommodating at that point, I would feel walked over and sold out. If I was to begin Accommodating more, I would then allow more conflicts to walk over me and I would begin to lose a sense of self-respect. I absolutely believe one should \emph{stir up a hornet's nest}, since shaking up problems sheds light on them, and allows those problems to be fixed. One should never have the mindset \emph{If you cannot make a man think as you do, make him do as you do}, since that attempts to force one's hand into doing something they disagree with, and cannot put legitimate effort into, will fork out even further issues, and lose even further trust with the person you are in conflict with since they cannot understand nor refuse to understand your point of view. Lastly, \emph{might overcomes right} is absolutely a false statement, since I perceive that as assaulting someone during an argument.

\end{document}
