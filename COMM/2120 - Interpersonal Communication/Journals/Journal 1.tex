\documentclass[12pt]{article}
\usepackage{times}
\usepackage{setspace}
\doublespacing
\usepackage[margin=1in]{geometry}
\begin{document}
\title{Reflection Journal 1}

%Choose one of the following and write a 1-page paper:

%Chapter 1 Journal

%Review the section titled “Characteristics of Competent Communication” (pgs. 21-23) then answer the following questions:

%    Out of the six characteristics of competent communication, which one would you identify as your single major strength in interpersonal communication? Cite two specific personal examples where you displayed this characteristic.
%   Out of the six characteristics of competent communication, which one would you identify as your one significant weakness? Give one specific example to illustrate why you feel this is a problem area for you.
%    How can you improve your weakness for face-to-face communication?
%    How can you improve your weakness as it relates to communication over social media?
\par
Out of the six characteristics of competent communication, I would say that \emph{self-monitoring} would be my single major strength. When mid conversation, I can read faces of people pretty easy to either display that I am boring them to pieces or if I am making them genuinely interested. When I read that I am not engaging them in conversation, or that I am beginning to sound stupid I just give up on the conversation and move on to the next one. I believe that this is important because it allows me to not waste energy on someone not willing or interested to receive it. The nice thing about Self Monitoring is it allows me to be adaptable in the conversation if I want to be, especially when it comes to generational gap conversations (Upper or Lower). In order to be adaptable within conversation however, you must first have the self-monitoring skill to be competent at the adaptable conversation. Thus perhaps in order to be a competent adaptive communicator, you must first have competent self-monitoring?
% Sarah when you edit this, we may have to break up Paragraphs even further because this assignment is confusing the shit out of me.
\par 

\end{document}
