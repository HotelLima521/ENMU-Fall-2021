\documentclass[12pt]{article}
\usepackage{times}
\usepackage{setspace}
\doublespacing
\usepackage[margin=1in]{geometry}
\begin{document}
\title{Reflection Journal 1}

%Choose one of the following and write a 1-page paper:

%Chapter 1 Journal

%Review the section titled “Characteristics of Competent Communication” (pgs. 21-23) then answer the following questions:

%    Out of the six characteristics of competent communication, which one would you identify as your single major strength in interpersonal communication? Cite two specific personal examples where you displayed this characteristic.
%   Out of the six characteristics of competent communication, which one would you identify as your one significant weakness? Give one specific example to illustrate why you feel this is a problem area for you.
%    How can you improve your weakness for face-to-face communication?
%    How can you improve your weakness as it relates to communication over social media?
\par
Out of the six characteristics of competent communication, I would say that \emph{self-monitoring} would be my single major strength. When mid conversation, I can read faces of people pretty easy to either display that I am boring them to pieces or if I am making them genuinely interested. When I read that I am not engaging them in conversation, or that I am beginning to sound stupid I just give up on the conversation and move on to the next one. I believe that this is important because it allows me to not waste energy on someone not willing or interested to receive it. The nice thing about Self Monitoring is it allows me to be adaptable in the conversation if I want to be, especially when it comes to generational gap conversations (Upper or Lower). In order to be adaptable within conversation however, you must first have the self-monitoring skill to be competent at the adaptable conversation. Thus perhaps in order to be a competent adaptive communicator, you must first have competent self-monitoring?
% Sarah when you edit this, we may have to break up Paragraphs even further because this assignment is confusing the shit out of me.
\par 
Face to face, the facial expressions and tone from the person of interest in the conversation is what becomes quite apparent. Them looking for an exit with their eyes, not being 100\% glued to the conversation, forgetting what we were talking about, it's obvious when you aren't sparking their interest. And honestly, I really don't think it has much to do with the person talking about the particular subject, I think it's the person of interest just doesn't care about that topic at that point in time at all. However, that can work to your favor since you can tell immediately if someone cares about the topic and you can either continue with the topic or just completely move on.
% Sarah, at this point I'm four whiskey sours deep and drunk off my ass, good luck with editing
\par
Online, self-monitoring is slightly more difficult depending on how much character the person of interest puts into the conversation via text. Their replies can demonstrate their emotions and reactions \emph{precisely}, or on the other hand, they have a very high difficulty putting their emotions and reactions into text. I have noticed this mostly with the Baby-Boomer generation, but also some in the Gen-Y (Millenials) and Gen-Z (Zoomer) generations. However, I absolutely love communicating like this since it requires the least amount of effort, you can do it from the comfort of your own home, and you can tell \emph{immediately} and \emph{effortlessly} if the person of interest is intrigued in the conversation, and in turn, you. In a really weird way, this is the most rewarding portion of a self-monitoring communication strength- the online reward portion.
\end{document}
