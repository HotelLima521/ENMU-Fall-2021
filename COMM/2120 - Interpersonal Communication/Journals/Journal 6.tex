\documentclass[12pt]{article}
\usepackage{times}
\usepackage{setspace}
\doublespacing
\usepackage[margin=1in]{geometry}
\begin{document}
\title{Reflection Journal 6}

% How has having a “nonnormative” name, “unusual” spelling of your name, or “unique” nickname impacted your identity?
\par
Many people spell my name as 'Derrick' instead of 'Derek'. This doesn't impact my identity too much since it's a minor spelling issue. My name is also quite common so many people know exactly how to pronounce it, and I haven't experienced an issue with anyone mispronouncing it. I do have friends that call me by various nicknames though, which coincidentally are an alternate version of my name. A lot of close female friends call me by 'Dirk' or 'Derk' as a short and cute way of grabbing my attention. Additionally, my friend in Helsinki calls me by the name 'D\"ork', which is an interesting Nordic variation of my nickname.
% How have people perceived you based solely on your name? (Made assumptions about your culture, creativity, punctuality, abilities, etc.)
\par
I don't think anyone has perceived anything significantly about my name. It is a pretty generic American name that quite a few people have independent of race or culture. I think it's pretty difficult to make assumptions about someone by just going off of their name too, at least with as inclusive as most first world countries are; or at least closer to urban locales. In fact, the only time someone has made assumptions about my culture was from my accent which has pretty heavy Minnesotan roots, which in turn has pretty heavy Scandinavian roots. Additionally, I have also been assumed to be Jewish because of my beard by some random homeless person at a grocery store. 
% When people mispronounce your name, how does that make you feel? Do you correct them? If so, how do you go about correcting them? If not, why don’t you correct them?
I haven't had an issue with people mispronouncing my name, however I have had a few people call me by an incorrect name. For example, my academic advisor is the most recent person to call me by a name other than Derek. He is stuck in the mindset that my name is 'Darrell'. I have corrected him a few times, but he continues to think my name is Darrell.
\par
% The chapter states, “Immigrants to the United States run the risk of being discriminated against simply because of their non-Anglo names.”

%    If you have made a name change, describe what led you to that decision.
%    If you have not made a name change, but know someone who has, ask them what led to their decision.
%    What impact does such a change have on identity?
My friend in New South Wales, Australia has changed her name from Alana to "Monty", albeit it isn't a legal name. The reason for the preferred name is her dad's last name is Montgomery and is called "Monty" for short. She chose to take over that nickname to feel closer to her dad since they have grown further apart, thus she does it as a memoir. Monty stated that the name change has no impact on her identity, and is just a preference.
\par
% The chapter states, “Some people regard unique names as distinctive.”

%    How is your name distinctive to your culture, whether it is considered unique or traditional to said culture?
%    How do you feel about having a name that is distinctive of your culture?
My name isn't very distinctive, at least in the modern era. My name is a shortened version of the name Diederik, which is a West Germanic version of the name Theodoric, which translates to People-Ruler. I feel indifferent about my name since it's not distinctive to my culture, the shortened name "Derek" is borrowed from the Dutch, similar to how the other form "Terry" is borrowed from the French's "Thierry". Derek stemming from 'doric' in Theodoric, and Terry/Thierry stemming from 'Theo'.
\end{document}
