\documentclass[12pt]{article}
\usepackage{times}
\usepackage{setspace}
\doublespacing
\usepackage[margin=1in]{geometry}
\begin{document}
\title{Reflection Journal 13}
%Defensiveness implies protecting ourselves from a perceived threat. The universal tendency is to try to “save face” by defending our presenting self when we perceive that it has been attacked by what social scientists call face-threatening acts. Frequently, this creates a climate that leads to a negative defensive spiral. However, not all responses are considered negative. Psychologist Jack Gibb isolated six types of defense-arousing communication and six contrasting behaviors that seem to reduce defensiveness (pgs. 371-377). For this assignment, you will do the following:

%    Identify two (2) different times when you perceived that you were under attack.
%        Briefly describe each situation, your response, and identify the Gibb categories you used.
%        Describe how you might have otherwise responded to create a more positive and supportive climate for both situations.
\par
During middle school, my sister and I moved full time into my step-dad's house. He is very strict and has an "I am always right even when I'm wrong" mentality. More and more chores were piled up on me and I found discrepancies in simple ways that I was treated versus my sister, such as when he made us hot chocolate, he split the packet and gave me a slight amount of powder and mostly water, whereas she had the vast majority of the packet. When I told my sister that all I have is colored hot water, she asked to try my hot chocolate, then acknowledged that mine was horrible. I also noticed how I wasn't allowed inside the kitchen after dinner to get anything to eat or drink, however she had free reign, and if I broke that rule I was punished (which the severity of many cases increased until I moved out). Many times, when I was trying to form a relationship with him, he would simply ignore me showing that I don't matter to him. Or, alternatively, he would just laugh at me. Initially I ignored it, but then it began to snowball on me and I began addressing it in an \emph{evaluation} defensive behavior when I began asking "Why do you always laugh at me when I say something?". I don't think I could have responded in a reasonably better way, especially due to my age at the time, and I don't think it would be appropriate for me to even do that.
\par
People tend to state when they get to understand my skill sets and in-depth knowledge about various subjects how I am \emph{so} much smarter than them. However, it has begun to bother me to a great extent; I know that they are far more knowledgeable in other subjects than I. For example, I have a friend that constantly states that she's stupid, however I watch her resolve a broad range of problems as she finds them. She knows \emph{how} to ask for help, \emph{who} to ask help from, and then executes. I've watched her repair electrical work in her house, completely resolve her husband's credit issues, buy five houses and rent most of them out, and turn around a failing 501c3. When she begins to downplay her skill sets compared to mine, I swiftly correct her and state that she is vastly more knowledgeable than I am in other areas and that I look up to her for that, which I would consider an \emph{empathetic} supportive behaviour. Again, this is not something I think can be improved upon since I believe it is at its best possible outcome.
%    Analyze your defensive behavior. In general, do you consider yourself a “defensive” person? Why or why not?
\par
I do not consider myself a defensive person, since the vast majority of the time a conflict arises, I believe that I am the one that has caused the issue and is upon my responsibility to resolve it. Only during the times where I am forced up against a wall such as the mental and physical abuse I received from my step dad, or the ignorance of my Principal in Elementary where it was my fault no matter the situation, I believe it has been engrained into my thinking pattern that "If there is an issue, it's my fault that it happened".
%    What topics tend to trigger a defensive response from you?
\par
Finances are the only topic I can think of that can trigger a defensive response from me if the other person is accusatory. However, if the other person shows that they are interested in helping me work through it, I show very supportive behavior in return, since I recognize the mess I have put myself in; I also recognize that it's my responsibility to get myself out of the mess.
%    Are there certain people with whom you become more defensive than normal?
\par
The only person I show more defensive behavior than normal with would be my step-dad, which should be understandable from the initial paragraph. For everything I have been through with him, being on defensive behavior by default is what makes me feel the most prepared for anything he throws at me.
\end{document}
