\documentclass[12pt]{article}
\usepackage{times}
\usepackage{setspace}
\doublespacing
\usepackage[margin=1in]{geometry}
\begin{document}
\title{Reflection Journal 11}
% Review the sections titled “Types of Friendships” (pgs. 292-294) and “Friendships, Gender, and Communication” (pgs. 294-297). Then choose at least two of your own friendships to discuss below:

% - Characterize your friendships according to the types discussed in these sections of the chapter (including same-sex, cross-sex, and/or friends with benefits if applicable).
\par
My first friend would be characterized as a cross-sex relationship in which one of us desires romance, most likely short term, task oriented, high disclosure, high obligation, and frequent contact. My second friend would be characterized as a cross-sex relationship which is strictly platonic, long term, maintenance oriented, high disclosure, high obligation, and infrequent contact. My third friend would be classified as a same-sex relationship, long term, maintenance oriented, low disclosure, low obligation, and infrequent contact.
% - What are the advantages of each type of friendship?
\par
The first friend, I feel much more attractive just being around them. She gives off an aura where I feel that her attractiveness makes me appear nearly as attractive just because I'm with her. Additionally, she is very sharp in our degree, as well as with mathematics. For my second friend, she's always inviting me to parties (at least when she lived here), and always wanted to bring me places to introduce me to people. For my third friend, he helps out constantly with programming information, and is consistently my drinking buddy whenever I am home.
% Review the section titled “Communication Rituals and Rules” (pg. 303).

% - What rituals does your family have? Are they centered around celebrations or part of everyday life?
\par
The only ritual I can think of with my family on my Mom's side is the whole family gathering for Christmas. We will converge in from different states, and from Canada (whether British Columbia, or Quebec), and meet up to visit each other for one day of the year and catch up. The traditions for my Dad's side of the family have all expired after the passing of my Grandmother who kept the family together for the most part. When my Grandmother passed, we no longer gathered together for Thanksgiving nor Christmas, or even Mexican food during anyone's birthdays. With her gone, it appears that no one on my Dad's side of the family wants to be around each other anymore.
% - What rules does your family have? Are these explicit or implicit? What happens when a rule is violated?
\par
I can't really think of many rules from either side of my family, with the exception of my step family on my Mom's side (my Mother's husband). When I lived there, we had to always take off our shoes in the house, complete our chores and complete our homework. The problem I experienced, which leads into rules being violated, was I had a major issue focusing on homework at home since I'd hit burnout so easily. In addition, the chores that he piled on added to the stress and it felt like I had a never ending amount of work I had to accomplish. When I wasn't able to finish a task I would be punished, which in rebellion I lashed out. From there, power would be cut to my bedroom \emph{which included heat during the winter}, locked out of the main house to prevent me from getting something to eat if I was hungry after dinner, or in an extreme case-- charging me causing me to trip over my mom laying on the floor, and breaking my arm.
% Review the section titled “Conversation and Conformity in the family” (pgs. 305-306).

% - What “family communication pattern” does your family illustrate?
\par
My family on my Mom's side would be Pluralistic. We communicate openly and often, especially if there is a problem that needs discussion. We work out, and suggest things to help the other get through the issue. On my Dad's side of the family, the communication is consensual since he is absolutely willing to listen to you make your case, though he is definitely the one that has the final say if something is needed. For my step family, they are completely Protective, albeit I would consider Authoritarian to describe it better.
% - Provide an example demonstrating how your family fits the communication pattern listed above.
\par
In elementary school and middle school, I was consistently getting in trouble with other kids. Between fighting to assault due to being bullied, I would be suspended often. Each time, my mom would explain to me she understands my issue and try to resist the perpetrators as best as I could while she tackles the larger issue which was the school administration (notably the Principle in elementary). She later explained after I was diagnosed with ADHD that some of the teachers in Elementary requested that I get diagnosed for ADHD and get prescribed medication to treat it, which my grandmother was completely against. I explained to my mom that I wish that she did follow through with the teacher's requests since I highly regard the problems I faced in Elementary to many social and mental health issues I face today.
\end{document}
