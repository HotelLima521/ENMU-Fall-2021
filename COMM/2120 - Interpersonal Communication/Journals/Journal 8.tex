\documentclass[12pt]{article}
\usepackage{times}
\usepackage{setspace}
\doublespacing
\usepackage[margin=1in]{geometry}
\begin{document}
\title{Reflection Journal 8}
\par
% - Copy an example of an original post and another person’s response (comment) to the post. What did the respondent say to indicate that they “heard” the original poster’s message?
Original post: "What Is te good setup for 4.3 v6 for all motor pump gas use 93 oct?" Response: "Unless your compression required 91/93 then I would just use 87. Its been proven that at optimized timing and afr the octane rating doesn't make any more power." Sub-Response: "he's asking how to build a good all motor 4.3 setup to run on 93" The respondent gave an accurate response to what the original poster was asking, which was how to build a high compression engine and gain as much power out of the engine as possible to work with a high-octane pump fuel, versus all of the other responses indicating not to use high-octane pump-fuel on a stock-compression engine.
\par
% - Copy an example of an original post and response (comment) where you perceived that the other person was not “listening” to what the original poster said. What did the respondent say to make you believe that the original poster was not heard?
Original post: "What Is te good setup for 4.3 v6 for all motor pump gas use 93 oct?" Response: "I use 87, occasionally 89", this response to the original post indicated that the responder didn't bother to understand what the original poster was asking. They just responded with effectively, "I use a low octane pump fuel", which indicates his engine is not built for high performance and is adding junk information to the post which will ultimately be glazed over.
\par
% - Do you feel that other people are sincerely listening to you when you interact with them via social media? Why or why not?
The vast majority of the time, I do not feel that people are sincerely listening to me when I interact with them on social media. I think it is because, especially in technical questions, people want to \emph{think} you're asking something else that they understand, or that you are trying to convey something else. I experienced this issue recently when asking an in-depth question about TLS certificates for a few internal websites behind my web proxy server for proper end-to-end encryption, which will allow full-functionality of the internal website, removing the breakages. The only time I feel that people are sincerely listening to me is when I am either A: in dire need of something mentally, or B: the information is broken down enough to where it's a light hearted conversation and fun.
\par
% - What suggestions do you have to help others, or yourself, improve when it comes to listening in conversations held via social media?
My suggestion for others, especially in the case of the first two paragraph's for the original poster would be first, proof read your question and ensure it's easily readable. The second would be ensure it's verbose enough but not \emph{too} much information-- that way the question isn't vague but also not an article, which would cause people to skip over reading it. To help myself is a bit tricky. It's difficult for me to self-police my listening skills in conversation on social media, since I feel like I am decent at understanding conversations on social media.
% *You can black out names for privacy if you wish.
\end{document}
