\documentclass[12pt]{article}
\usepackage{times}
\usepackage{setspace}
\doublespacing
\usepackage[margin=1in]{geometry}
\begin{document}
\title{First Amendment}
\par
For Mr. Hatzfield's speech, I enjoyed his portion where he stated that the United States is the greatest society known to man. Where it can be arguable that the greatness is fading in comparison to other countries, the US does remain a shining example of such. I wanted to argue that Rome would have been the greatest society, but they maintained slavery for hundreds of years, whereas the US ended it within the first century. Additionally, upon the riot section, I did not like how Mr. Hatzfield did not mention that yes, we have a freedom of speech-- however that does not make us free from consequences, which pushes hateful speech away (e.g. the Westboro Baptist church preaches hate speech, faces consequences, yet still has their freedom of speech). I really liked his quote from Stella Payton that said "Bloom where you are planted", since we have the freedom of choosing our own success versus communist countries which dictates our career and our movement, or other countries that dictate our success dependent on our cast or sex.
\par
I was confused at first when Mr. Fuqua's speech was starting, and wasn't too sure when to exactly begin listening. I wish that he had text up to follow along like Mr. Hatzfield's speech, since I was able to far more easily grasp what he was talking about. I did absolutely love his energy. I \emph{really} liked how he stated how students are utilizing volunteerism more to make small changes, which we feel that we can actually control the outcome. Additionally, I liked how he stated how we recognize our phones are here, however communication over our phones don't feel as \emph{real}.
\par
I did not like how difficult it was to understand Mr. Okoli's speech, however that isn't his fault since he is not a native speaker. It was very difficult to get a proper grasp of what he was talking about. Perhaps if he had text in his presentation to follow along, it would make it far easier. I was shocked when Mr. Okoli stated that six students were arrested for posting offensive posts on Facebook, and made me think about he we take our freedom for granted. When he stated that slander is not stated under the freedom of speech, and anything that can intimidate, hurt, or disturb the peace from another student can be held pretty open for interpretation, and that was also pretty scary to me.
\end{document}
