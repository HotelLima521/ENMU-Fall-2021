\documentclass[12pt]{article}    
\usepackage{times}    
\usepackage{setspace}    
\doublespacing    
\usepackage[margin=1in]{geometry}    
    
\usepackage{rotating}    
\usepackage{multirow}    
\usepackage{lineno}    
\usepackage{fancyhdr}    
\pagestyle{fancy}    
\lhead{}    
\lfoot{}    
\cfoot{}    
\rfoot{}    
\renewcommand{\headrulewidth}{0pt}    
\renewcommand{\footrulewidth}{0pt}    
\setlength\headsep{0.333in}    
\newcommand{\bibent}{\noindent \hangindent 40pt}    
\newenvironment{workscited}{\newpage \begin{center} Works Cited \end{center}}{\newpage }    
    
    
    
    
\begin{document}    
\title{Position Paper 2}

%The purpose of this assignment is for each student to demonstrate knowledge of topics covered thus far in Interpersonal Communication. The position paper gives each student the opportunity to apply concepts from the text, lectures, and classroom discussions to real life examples and experiences. You will have two of these due during the course of the semester. You may choose your own topic about which to write. Each paper will be graded based on the originality and thoroughness of content, your ability to articulate and defend your chosen position, and the use of vivid and appropriate examples.
%
%Each paper must be 2-4 pages in length (this means at least two full pages and no more than four), typed in 12-point Times New Roman with one inch margins. Each paper should include your name and be stapled in the upper left-hand corner.
%
%Topic Selection: Please choose one of the following topics to write your paper about. Keep in mind, the point of this paper is to choose a topic that you have a strong opinion about and about which you can provide suitable examples.
%
%    In your opinion, what is the role of research and theory in the study of interpersonal communication? Which do you feel is more important and why?
%  -  Defend or refute (prove false) that “interpersonal communication is constantly in motion and changing over time.”
%  -  Defend or refute that “communication is irreversible.”
%  -  Defend or refute that “interpersonal communication is not a necessary part of maintaining our public self.”
%  -  Does culture, gender, or personality have the most significant impact upon our perceptions in interpersonal communication?

\end{document}
