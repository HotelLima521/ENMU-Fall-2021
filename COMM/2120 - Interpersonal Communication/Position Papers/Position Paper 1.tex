\documentclass[12pt]{article}
\usepackage{times}
\usepackage{setspace}
\doublespacing
\usepackage[margin=1in]{geometry}
\begin{document}
\title{Position Paper 1}
% The purpose of this assignment is for each student to demonstrate knowledge of topics covered thus far in Interpersonal Communication. The position paper gives each student the opportunity to apply concepts from the text, lectures, and classroom discussions to real life examples and experiences. You will have two of these due during the course of the semester. You may choose your own topic about which to write. Each paper will be graded based on the originality and thoroughness of content, your ability to articulate and defend your chosen position, and the use of vivid and appropriate examples.

% Each paper must be 2-4 pages in length (this means at least two full pages and no more than four), typed in 12-point Times New Roman with one inch margins. Each paper should include your name and be stapled in the upper left-hand corner.

% Topic Selection: Please choose one of the following topics to write your paper about. Keep in mind, the point of this paper is to choose a topic that you have a strong opinion about and about which you can provide suitable examples.

% In your opinion, what is the role of research and theory in the study of interpersonal communication? Which do you feel is more important and why?
% >> Defend or refute (prove false) that “interpersonal communication is constantly in motion and changing over time.”
%    Defend or refute that “communication is irreversible.”
%    Defend or refute that “interpersonal communication is not a necessary part of maintaining our public self.”
%    Does culture, gender, or personality have the most significant impact upon our perceptions in interpersonal communication?
\par
% Intro
I personally believe that Interpersonal Communication is constantly in motion and changing over time. The more integrating cultures become, the more we begin to understand each other, allowing us to understand when one thing we say can be quite insulting or a really sweet compliment. For example, my Grandparents grew up prior to both Women's suffrage and the Civil Rights Movement. Their perspectives on  both Black people and Women in politics as well as the work force were quite a bit different in comparison to my generation. As time evolves, communication \emph{must} move with it. I will now list some important cases in point.
\par
% English Now vs Then
English today is vastly different in comparison to the English of yesterday. We have seen accents in the United States change wildly due to early colonists attempting to mimic the Aristrocatic English to fit in better. At some point the accent stuck, leaving us with a relatively unique accent. The southern accent being the closest mimic of the English accent due to Slave Owners being in a higher economic status in comparison to most Northern Americans. Coincidentally, during the United States Civil War, England directly supported the South which could have something to do with the accents as well since there would be interest in the livelihood of various cotton plantations that England sourced from. 
\par
% Romance Languages
Romance languages have also significantly changed. Romance languages, stemming from Latin, have split off into various other sub-languages such as: Spanish, Italian, French, Portuguese, Romanian, Catalan, amongst a few other less spoken ones. After the fall of the Roman Empire, I am assuming that each of these languages evolved to fit in with their people. Because the Roman Empire lasted between 27 B.C. to 476 A.D., that was far more than enough time for many sects of the language to evolve and split.
\par
% Germanic Languages
Like Romance languages, Germanic languages have morphed quite a bit too. All Germanic languages are thought to have been decended from a proto-germanic language
\par
% Culmination of Languages combining into one (e.g. Spanglish)
\par
% Pirahan People
\par
% Hyroglyphics
\par
% Outro
\end{document}
