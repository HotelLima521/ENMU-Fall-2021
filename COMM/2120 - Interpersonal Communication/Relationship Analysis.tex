\documentclass[12pt]{article}
\usepackage{times}
\usepackage{setspace}
\doublespacing
\usepackage[margin=1in]{geometry}
\begin{document}
\title{Reflection Journal 7}

% You will view a movie of your choice and write a 5-6 page, 12-pt Times New Roman or Courier New font, double-spaced typed, detailed paper that applies interpersonal communication concepts learned in the course to situations revealed in the movie. (We will discuss this project AND various films in class; some examples include “The Way,” “The Intouchables,” “The Kid with a Bike,” “Love in Another Language,” to name but a few). In this paper, you will discuss what you’ve read in your textbook and what we’ve talked about in class in relation to your film. Do not give the story plot, but demonstrate your understanding of interpersonal communication, relationships and concepts of the film. You will need to refer to situations in the film to guide your discussion of interpersonal theory, but realize that I’m looking for an analysis paper of interpersonal communication from the film, not a film review.
 
% The purpose of this assignment is to allow you to identify interpersonal communication concepts and provide a detailed analysis of those concepts and how they are demonstrated in the film. You can also add an analysis of how communication could have been improved and what might have occurred if certain behavior demonstrated in the film had continued rather than stopped or stopped rather than continued. You are encouraged to concisely apply your own observations of life and situations to your analysis (such as situations that you have observed that are similar as one(s) revealed in the film, the similarities and differences with the situations and how they resolved, and the analysis of both situations to the concept/theory you are applying and discussing in that section of the paper).

% This paper will be graded on how well it meets the above criteria, editing for a direct and in-depth analysis, grammatical accuracy, expression of ideas, and synthesis of the material. The paper must be typed, double-spaced, and no less than 5 full pages (only 1-inch margins on each page)

% Intro
% Introduce character
\par
The fact that Jesse and Mike are standing at a relatively close proximity to each
other (about 4 feet) gives an indication to the viewer that these two individuals
are relatively familiar with each other. The proximity in which an individual
place themselves to another individual is a type of non-verbal communication,
where a shorter distance generally indicates a more intimate relationship. This
makes sense because in the general public, at least in the United States, romantic
partners tend to remain in an intimate distance of one another (skin contact
to 18 inches) whereas strangers tend to generously space themselves (greater
than 12 feet). Sure enough, this idea is reinforced when the two begin to have a
conversation reflecting on their past choices.
\par
Viewers watching El Camino who are unfamiliar with the Breaking Bad series
will soon find themselves confused as the movie cuts to a scene where we now
view a dirty, tattered, and bearded Jesse frantically driving down a road. El
Camino introduces Jesse as a calm, normal-looking guy and now, suddenly, he
appears as a completely different person. An unfamiliar viewer will immediately
begin attempting to make sense of the scene, wondering things like “what did
this guy do,” “is he crazy,” and “where is he going?” These questions tie into a
common tendency of perceiving others, in this case, making snap judgements.
The audience makes a snap judgement about Jesse given that our first impression
of him was an innocent looking guy, which actually ties into another tendency,
one where we cling to first impressions.
\par
The next scene displays a tattered and weakened Jesse leaning against the
front door of a house, which is revealed to be the house of Jesse’s two friends,
Badger and Skinny Pete. After creaking the door open, Badger and Skinny
Pete are struck with disbelief when they realize the face of this tattered man
belongs to their friend, Jesse. This scene was iconic for two reasons: it was a
pivotal moment in the movie’s plot and the audience was able to empathize with
1Badger and Skinny Pete at the sight of a friend in desperate need of assistance.
Empathy is the ability to re-create another person’s perspective to experience the
world from his or her point of view and is one of many factors of interpersonal
communication. When I saw Badger and Skinny Pete eagerly helping Jesse, it
instantly reminded me of the time I let one of my close friends live with me
for a while because they were desperate to get out of their unhealthy living
situation. Being able to interact with another person and constructively discuss
their problems can help strengthen the relationship between yourself and that
person.
\par
In the following scene, now that all the tension from the previous night has
diminished, we are presented with a mildly comical scene of Badger and Skinny
Pete attempting to have Jesse answer their questions regarding his whereabouts.
Jesse, still in a mild state of shock, is only able to reply in short, vague, and
generally unhelpful sentences. Jesse’s response consisted of indirect, barely
succinct, and informal sentences, which is a type of verbal communication style.
Verbal communication styles are based on three factors: how direct the message
is, how elaborate the message is, and its level of formality. Badger and Skinny
Pete were expecting a direct, elaborate, and formal response from Jesse, but
received the exact opposite, which is why the scene was somewhat comical.
Although Jesse was in no proper state of mind to deliver a clear message, the
uncertainty felt by Badger and Skinny Pete could have been eliminated much
earlier should Jesse have been able to speak clearly and fluently.
\par
Once Jesse explains that his kidnappers were killed and he is currently on the
run from the police due to dealing meth, Skinny Pete was done silently listening
and took control of the situation. In this scene, Skinny Pete conveyed two
types of listening responses: silent listening and advising. At first, Skinny Pete
remained silent and attentive to Jesse’s concerns without offering verbal feedback.
After Jesse was finished speaking, Skinny Pete switched to an advising listening
response and came up with a plan that temporarily relieved Jesse of his problems,
giving Jesse time to think of a more permanent solution. Silent listening is a
listening response that is more reflective and less directive of a person’s problem.
The opposite of silent listening is advising, in which it is less reflective and more
directive of a person’s problem. Skinny Pete’s decision to listening to all of what
Jesse had to say allowed him to come up with a solution that greatly assisted
his friend.
\par
At this point, Jesse is using Skinny Pete’s car to drive around town to get
his affairs in order before fleeing New Mexico. While driving down the city,
Jesse is forced to pull over as a brigade of squad cars light up his rear-view
mirror. Jesse becomes immobile out of fear and reluctantly comes to terms
with the thought of getting arrested. To Jesse’s surprise, the squad cars drive
past him, presumably responding to a different crime. What Jesse displayed
in this scene was a debilitative emotion: an emotion that hinders or prevents
performance. Interpersonal communication explains that the interpretation of
events determines feelings. In this case, Jesse interpreted the squad cars were
2after him and the thought of getting arrested struck Jesse with an overwhelming
sense of fear, preventing him from thinking of an escape plan. We as humans have
the ability to turn debilitative emotions into facilitative emotions, or emotions
that contribute to effective functioning. We are able to do this by changing
unproductive interpretations of events. If Jesse had not jumped to conclusions in
his head, he would have spared himself immobilization and an adrenaline spike.
\par
Next, we cut to a flashback of Jesse’s time as a slave to a rival drug gang and
white supremacist group. We see the gang leader’s son, Todd, left in charge
of watching Jesse while they were away conducting business. Todd, being
slightly less crazy than his father, treated Jesse like a dog; he kept Jesse in
an underground cage where he would give him food, water, and an occasional
cigarette. Throughout the flashback we see several instances of Jesse putting up
with Todd’s degrading behaviors. This flashback displays what is known as a
high-power distance. A high-power distance is when the members of a society
accept a massively unequal distribution of power, in this case, Jesse is forced to
obey Todd’s commands. In interpersonal communication, the greater the power
distance, the more affected the transactional communication model becomes.
Once the flashback is over, we find Jesse parked under a bridge tuning the
car’s radio trying to find information about the police’s attempts to capture
him. Unexpectedly, Jesse is in disbelief upon hearing news of the death of his
multi-million-dollar, meth-dealing partner, Walter White. The method in which
Jesse heard this news, via radio, can be considered a lean message. Messages
sent via email, text, letters, and online posts are examples of lean messages
whereas messages delivered face-to-face are considered richer messages.
\par
We now cut to another flashback during Jesse’s days as Todd’s slave, this time, in
the apartment of one of Todd’s murder victims. This scene is disturbing in which
Todd is heating up tomato soup on his victim’s stove top while simultaneously
justifying why he had to kill her. Once the soup is ready, Todd seats Jesse down on
a small table barely big enough for the both of them, serves both of them tomato
soup, and begins to eat. Jesse, however, remains motionless, staring indefinitely
at his soup. This scene portrays an example of non-verbal communication,
specifically body language. Jesse is presumably disgusted by Todd’s presence,
especially after hearing his excuse for killing an innocent woman. Unfortunately,
since he is in no position to argue, he remains speechless and motionless. Body
language is one of several forms of non-verbal communication that reveal the true
thoughts and intentions of a person. It is physically impossible not to convey a
non-verbal message. In addition to his lack of movement, another non-verbal
message delivered is Jesse’s clothing. While Todd is perfectly groomed and
well-kempt, Jesse is unkempt with messy hair, a dirty face, and torn, ragged
clothing. Clothing alone conveys factors such as economic level, social position,
trustworthiness, and education level.
\par
This next moment is also an iconic scene in the movie. We now find present Jesse
at Todd’s apartment where he is rumored to have a hidden stash of three-hundred
thousand dollars. As Jesse is frantically tearing up the apartment in search of
3the money, two other criminals, one by the name of Neil, also appear at Todd’s
apartment. After the three criminals found themselves in a stalemate, they
agreed to split the money in thirds and be on their way; the criminals reached
what is known as a compromise. A compromise is one of several methods of
resolving a conflict, specifically, a compromise is a lose-lose situation in which
all disputing parties give up a portion of what they want in order to reach an
agreement. Each criminal wanted the money all to themselves, and so it is a
compromise since they had to settle for a third each.
\par
Though there remain several more examples of interpersonal communication
in El Camino, the examples provided should suffice in proving interpersonal
communication can be found in not only our day-to-day life, but also in cinematic
universes. This leads me to believe that one reason movies are popular is because
even though most are based on fictional events and characters, we are still
able to take away lessons from films. In the case of this analysis, interpersonal
communication is utilized to create meaningful interactions between characters
and the audience.
%\par
%The beginning scene shows <character> walking the streets with his only belongings, searching for work, displaying he is homeless and lower class.

% Heavy usage of mediated communication (TV's)
%<black character> is complaining about steel worker companies and <character> after listening provides some advice, which displays the two types of lisnening responses: \emph{silent listening and advising}. Initially <character> held is silence and listened attentively to <black character>'s statement until <black character> was finished. From there, <Character> switched gears into an advising listening response which <black character> got even further aggressive to the steel worker companies, however seemed to also be slightly relieved.
%\par
%\par


\end{document}
